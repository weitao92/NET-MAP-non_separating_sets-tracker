\documentclass[article,dvisp]{amsart} 

\usepackage{graphicx}
\usepackage{amssymb}
\usepackage{epstopdf}
\usepackage{pspicture}
\usepackage{pstricks} 
\usepackage{pst-all}

\usepackage{tabularx}
\usepackage{bigstrut}
\usepackage{hyperref}

\usepackage{dsfont}
\usepackage{multirow}
\usepackage{amsfonts}
\usepackage{amssymb}
\usepackage{amsthm}
\usepackage{newlfont}
\usepackage{graphics} 
\usepackage{graphicx}
\usepackage{amscd}        
\usepackage[all]{xy}
\usepackage{pspicture}
\usepackage{pstricks} 



\renewcommand{\H}{\mathcal{H}}
%\renewcommand{\d}{\textbf{d}}
\newcommand{\co}{\mathbb{C}}
\renewcommand{\a}{\mathcal{A}}
\renewcommand{\b}{\mathcal{B}}
\renewcommand{\r}{\mathcal{R}}
\renewcommand{\u}{\mathcal{U}}
\renewcommand{\l}{\mathcal{L}}
\newcommand{\x}{\mathcal{X}}
\newcommand{\y}{\mathcal{Y}}

\renewcommand{\t}{\mathbb{T}}
\newcommand{\ba}{\boldsymbol{\alpha}}

\def\a{\textbf{a}}
\def\b{\textbf{b}}
\def\c{\textbf{c}}
\def\d{\textbf{d}}

\def\S{\mathcal{S}}
\def\M{\mathcal{M}}
\def\R{\mathbb{R}}
\def\RR{\mathcal{R}}
\def\P{\mathbb{P}}
\def\Q{\widehat{\mathbb{Q}}}
\def\H{\mathbb{H}}
\def\C{\mathbb{C}}
\def\O{\Omega}
\def\U{\mathcal{U}}
\def\Y{\mathcal{Y}}
\def\CC{\overline{\mathbb{C}}}

\newcommand{\n}{\mathbb{N}}
\newcommand{\ov}{\overline}
\newcommand{\He}{\mathcal{H}_{L_{1},\epsilon}}
\newcommand{\h}{\mathbb{M}_{n}(H(\r))}
\renewcommand{\ss}{\mathbb{M}_{1}(H(\r))}
\newcommand{\sss}{\mathbb{M}_{2}(H(\r))}
\newcommand{\st}{\mathbb{M}_{3}(H(\r))}
\renewcommand{\k}{\mathbb{K}}
\newcommand\Tau{\mathcal{T}}

\newcommand{\zp}{\mathbb{Z}/p\mathbb{Z}\oplus\mathbb{Z}/p\mathbb{Z}}
\newcommand{\ds}{\displaystyle}
\def\Z{\mathbb{Z}}


\newtheorem{thm}{Theorem}[section]
\theoremstyle{definition}
\newtheorem{defn}[thm]{Definition} 
\newtheorem{ex}[thm]{Example}
\newtheorem{lem}[thm]{Lemma}
\newtheorem{cor}[thm]{Corollary}
\newtheorem{prop}[thm]{Proposition}
\newtheorem{rem}[thm]{Remark}
\theoremstyle{remark}
\newtheorem{clm}[thm]{Claim}
\numberwithin{equation}{section}

%%%%%%%%%%%%%%%%%%%%%%%%%%%%%%%%%
\theoremstyle{lemma}
\newtheorem{lema}{Lemma}[section]
\newenvironment{Lemma}{\begin{lema}}{\end{lema}}

\newtheorem{coro}[lema]{Corollary}
\newenvironment{Corollary}{\begin{coro}}{\end{coro}}

\newtheorem{teo}[lema]{Theorem}
\newenvironment{Theorem}{\begin{teo}}{\end{teo}}

\newtheorem{defi}[lema]{Definition}
\newenvironment{Definition}{\begin{defi}}{\end{defi}}

\newtheorem{afin}[lema]{Claim}
\newenvironment{afi}{\begin{afin}}{\end{afin}}

\newtheorem{pro}[lema]{Proposition}
\newenvironment{Proposition}{\begin{pro}}{\end{pro}}
\newtheorem{remm}[lema]{Remark}
%%%%%%%%%%%%%%%%%%%%%%%%%%%%%%%%%

\newcommand{\Frac}{\displaystyle \frac}
\newcommand{\Sqrt}{\displaystyle \sqrt}


\begin{document}
\title[On the Classification of Nonseparating Substes]{On the Classification of Nonseparating Subsets}
\author[Weitao Li and Edgar Saenz]{Weitao Li and Edgar Saenz}
\address{Department of Mathematics,
Virginia Tech,
Blacksburg, VA 24061, U.S.A.}
\email{weitao92@vt.edu}
\address{Department of Mathematics,
Virginia Tech,
Blacksburg, VA 24061, U.S.A.}
\email{easaenzn@vt.edu}
\begin{abstract} 
In this paper, we investigate the existence, some general properties and the classification of nonseparating subsets of finite Abelian groups $A$ generated by two elements with exponent at most 10 such that $A/2A\cong \Z_{2}\oplus\Z_{2}$.\end{abstract}
\maketitle
\section{Introduction}
%A Thurston map $f$ is called \begin{em}nearly Euclidean\end{em} if its local degree at each critical point is 2 and it has exactly four postcritical points. 
A Thurston map $f:(S^{2},P_{f})\to(S^{2},P_{f})$ with postcritical set $P_{f}$ is called NET map if each critical point is simple and $|P_{f}|=4$. NET maps are simple generalizations of rational Latt\`{e}s maps. From Lemma 1.3 of \cite{CFPP}, it follows that every NET map $f$ has the property that $f^{-1}(P_{f})$ contains exactly four points which are not critical points. As a consequence, NET maps lift to maps of tori as follows.
\begin{thm}\label{teo:31} Let $f$ be a Thurston map. Then $f$ is nearly Euclidean if and only if there exist branched covering maps $p_{1}:T_{1}\to S^2$ and $p_{2}:T_{2}\to S^2$ with degree 2 from the tori $T_{1}$ and $T_{2}$ to $S^2$ such that the set of branch of $p_{2} $ is $P_{f}$ and there exists a continuous map $\tilde{f}: T_{1}\to T_{2}$ such that $p_{2}\circ\tilde{f}=f\circ p_{1}$.\end{thm}

For $j\in\{1,2\}$, let $P_{j}\subset S^2$ be the set of branched points of $p_{j}$  and let $q_{j}:\R^2\to T_{j}$ be a universal covering map. The map $p_{j}\circ q_{j}:\R^2\to S^2$ is a regular branched covering map whose local degree at every ramification point is $2$. We can choose $q_{j}$ so that $\Gamma_{j}$ (the set of deck transformations of $p_{j}\circ q_{j}$) is generated by the set of all Euclidean rotations of order $2$ about the points of $\Lambda_{j}$ (the set of ramification points of $p_{j}\circ q_{j}$). We may normalize so that $0\in\Lambda_{j}$. It follows that $\Lambda_{j}$ is a lattice in $\R^2$ and that $\Gamma_{j}=\{x\mapsto2\lambda\pm x:\lambda\in\Lambda_{j}\}$. Furthermore, we can choose $q_{1}$ so that $\tilde{f}$ lifts to the identity map. Thus, $\Lambda_{1}\subseteq\Lambda_{2}$ and $\Gamma_{1}\subseteq\Gamma_{2}$. So we obtain the standard commutative diagram
$$\begin{CD}
\Lambda_{1}@>{i_{c}}>>\Lambda_{2}\\
{i_{c}}@VVV      @VVV{i_{c}}\\
\R^2@>id>>\R^2\\
{q_{1}}@VVV      @VVV{q_{2}}\\
T_{1}@>{\tilde{f}}>>T_{2}\\
{p_{1}}@VVV      @VVV{p_{2}}\\
S^2@>f>>S^2
\end{CD}$$\\
where $id:\R^2\to\R^2$ is the identity map and the maps from $\Lambda_{1}$ to $\Lambda_{2}$ are inclusion maps. The standard commutative diagram implies that $\R^2/\Gamma_{1}$ and $\R^2/\Gamma_{2}$ are both identified with $S^{2}$. It is furthermore true that $p_{2}$ branches over $P_{f}$; i.e., $P_{2}=P_{f}$.\vspace{.1in}\\
\textbf{Topological Property.} Let $f$ be a NET map. A simple closed curve in $S^{2}\setminus P_{f}$ is $peripheral$ if it can be shrunk to a point in $P_{f}$. A simple closed curve in $S^{2}\setminus P_{f}$ is $essential$ if it is neither null homotopic nor peripheral in $S^{2}\setminus P_{f}$. We would like to know when a NET map $f$ verifies the following topological property.

Property($\Delta$): Every connected component of $f^{-1}(\delta)$ is inessential for all simple closed curve $\delta$ in $S^{2}\setminus P_{f}$.

In terms of Teichm\"uller theory, this property is equivalent to say that the Teichm\"uller map induced by the map $f$ is constant \cite{BEKP}. Roughly speaking, if a NET map $f$ verifies the property ($\Delta$), then we may regard $f$ as a rational map on the Riemann sphere.\vspace{.1in}\\
\textbf{Basic Group Situation.} Let $A$ be a finite abelian group. Let $H$ be a subset of $A$ which is the disjoint union of four pairs $\{\pm h_{1}\},$ $\{\pm h_{2}\},$ $\{\pm h_{3}\},$ $\{\pm h_{4}\}$. Let $B$ be a subgroup such that $A/B$ is cyclic, and let $a$ be an element of $A$ so that $a+B$ generates $A/B$. Let $n$ be the order of $A/B$. For each $k\in\{1,2,3,4\}$ there exists a unique integer $c$ with $0\leq c\leq n/2$ such that $(ca+B)\cap\{\pm h_{k}\}\neq\emptyset$.
Let $c_{1},c_{2},c_{3},c_{4}$ be these four integers ordered so that $0\leq c_{1}\leq c_{2}\leq c_{3}\leq c_{4}$. These four numbers are called the \begin{em}coset numbers\end{em} for $H$ relative to $B$ and the generator $a+B$ of $A/B$.

We are particularly interested in the coset numbers when $A=\Lambda_{2}/2\Lambda_{1}$, where $\Lambda_{1}$ and $\Lambda_{2}$ are the lattices given in the standard commutative diagram. If $\lambda, \mu$ form a basis of $\Lambda_{2}$, then $B=\langle \lambda+2\Lambda_{1}\rangle$ is a cyclic subgroup of $A$ so that $A/B$ is cyclic and $\mu+2\lambda_{1}$ is an element of $A$ whose image in $A/B$ generates $A/B$. Since $q^{-1}_{1}(p^{-1}_{1}(P_{2}))\subseteq \Lambda_{2}$, we may speak of coset numbers for $H=p^{-1}_{1}(P_{2})$. The advantage of dealing with NET maps is that given $\delta$, an essential simple closed curve in $S^{2}\setminus P_{2}$, it is possible to count the number of essential components in $f^{-1}(\delta)$ in terms of the coset number for $H=p_{1}^{-1}(P_{2})$. The following result can be found in \cite{CFPP}.

\begin{thm}\label{teo:2} Let $\delta$ be an essential simple closed curve in $S^{2}\setminus P_{2}$ with slope p/q with respect to the basis $(\lambda_{2},\mu_{2})$ of $\Lambda_{2}$. Let $\lambda=q\lambda_{2}+p\mu_{2}$. Choose $\mu\in\lambda_{2}$ so that $(\lambda, \mu)$ is a basis for $\Lambda_{2}$. Let $c_{1},c_{2},c_{3},c_{4}$ be the coset numbers for the elements of $p^{-1}_{1}(P_{2})$. Then, the number of essential components in $f^{-1}(\delta)$ is $c_{3}-c_{2}$.\end{thm}

The preceding theorem was the motivation of the following definition. In the basic group situation, $H$ is a $nonseparating$ subset of $A$ if and only if $c_{2}=c_{3}$ for every choice of $B$ and $a$. As an immediate consequence we get an algebraic formulation of the property ($\Delta$).

\begin{thm}\label{teo:3} A NET map $f$ verifies the property ($\Delta$) if and only if $p^{-1}_{1}(P_{2})$ is a nonseparating subset of the group $A=\Lambda_{2}/2\Lambda_{1}$.\end{thm}

In the Buff-Epstein-Koch-Pilgrim paper \cite{BEKP}, they characterize when the induced Teichm\"uller map of a Thurston map is constant. Unfortunately, checking whether or not the induced Teichm\"uller map is constant is very difficult primarily because there are infinitely many essential curves to
consider. However, in the particular case of NET maps, Theorem \ref{teo:3} provides an algebraic characterization of those NET maps whose induced maps on Teichm\"uller space are constant. This characterization reduces to the existence of nonseparating subsets for finite Abelian groups generated by two elements. Our work is focused on the existence and classification of nonseparating subsets and the number of Hurwitz classes of nonseparating subsets for groups with exponents at most 10.

%%%%%%%%%%%%%%%%%%%%%%%%%%%%%%%%%%%%%%%%%%%%%%%%%%%%%%%%%%%%%%%%%%%%%%%%%%%%%%%%%%%%%%%%%%%%%%%%%%%%%%%%%%%%%%%%%%%%%%%%%%%%%%%%%%%%%%%%%%%%%%%%%%%%%%%%%%%%%%%%%%%%%%%%%%%%%%%%%%%%%%%%%%%%%%%%%%%%%%%%%%%%
\section{Definitions and Known Results}

In this section, we first review the definitions and facts on $coset\ numbers$ and $nonseparating\ sets$. Then we will state several well-known facts about NET map and nonseparating set.

Let $A$ be a finite abelian group. Let $H$ be a subset of $A$ which is the disjoint union of four pairs $\{\pm h_{1}\},$ $\{\pm h_{2}\},$ $\{\pm h_{3}\},$ $\{\pm h_{4}\}$. (It is possible that $h_{i}=-h_{i}$.) Let $B$ be a subgroup such that $A/B$ is cyclic, and let $a$ be an element of $A$ so that $a+B$ generates $A/B$. Let $n$ be the order of $A/B$. For each $k\in\{1,2,3,4\}$ there exists a unique integer $c$ with $0\leq c\leq n/2$ such that $(ca+B)\cap\{\pm h_{k}\}\neq\emptyset$.
Let $c_{1},c_{2},c_{3},c_{4}$ be these four integers ordered so that $0\leq c_{1}\leq c_{2}\leq c_{3}\leq c_{4}$. These four numbers are called \begin{em}coset numbers\end{em} for $H$ relative to $B$ and the generator $a+B$ of $A/B$.

Let $A$ be a finite Abelian group. A subset $H$ of $A$ is called \begin{em}nonseparating\end{em} if and only if it satisfies the following conditions:

\begin{itemize}
\item $H$ is a disjoint union of the form $H=H_{1}\coprod H_{2}\coprod H_{3}\coprod H_{4}$, where each $H_{i}$ has the form $H_{i}=\{\pm h_{i}\}$. (It is possible that $h_{i}=-h_{i}$.)                          
\item Let $B$ be a cyclic subgroup of $A$ such that $A/B$ is cyclic. Let  $c_{1},c_{2},c_{3},c_{4}$ be the coset numbers for $H$ relative to $B$ and some generator of $A/B$. The main condition is that $c_{2}=c_{3}$ for every such choice of $B$ and generator of $A/B$.
\end{itemize}

The next two lemmas provide ways to produce nonseparating subsets from known ones from lower order. For details of the proofs, see Section 10 of \cite{CFPP}.

\begin{lem}\label{lem:55} Let $A$ be a finite Abelian group, and let $H$ be a nonseparating subset of $A$. If $\varphi:A\to A$ is a group automorphism and $h$ is an element of order 2 in $A$,  then $\varphi(H)+h$ is a nonseparating subset of $A$.\end{lem} 

\begin{lem}\label{lem:56} If $A$ is a finite Abelian group and if $A'$ is a subgroup of $A$, then every subset of $A'$ which is nonseparating for $A'$ is nonseparating for $A$. \end{lem}

The next lemma shows the converse of Lemma \ref{lem:56}. For details of the proof, see Appendix A of \cite{S}.

\begin{lem}\label{lem:a05} Let $A$ be a finite Abelian group generated by two elements and let $A'$ be a subgroup of $A$. If $H$ is a subset of $A'$ which is nonseparating for $A$, then $H$ is nonseparating for $A'$.\end{lem}

Based on lemma \ref{lem:55} and lemma \ref{lem:56}, given a finite Abelian group generated by two elements, we define an equivalence relation $\sim$ on the collection of nonseparating subsets of $A$ as follows.

\begin{defn}\label{def:90} Let $H_{1}$, $H_{2}$ be two nonseparating subsets of $A$. We say that $H_{1}$ is related to $H_{2}$ and write $H_{1}\sim H_{2}$ if and only if there exists $\varphi$ an automorphism of $A$ and an element $\tau\in A$ with $2\tau=0$ such that $H_2 = \varphi(H_1)+\tau$. The equivalence classes under this equivalence relation will be called in this paper $Hurwitz$ classes of nonseparating subsets.\end{defn}

Hurwitz classes of nonseparating subsets could be empty. Below is a result that supports this claim. For details of the proofs, see Section 10 of \cite{CFPP}.
\begin{thm}\label{thm:21} Let $A$ be a finite Abelian group such that $A/2A\cong Z_{2}\oplus\Z_{2}$ and $2A$ is a cyclic group of odd order. Then $A$ does not contain a nonseparating subset.
\end{thm}
\newpage
%%%%%%%%%%%%%%%%%%%%%%%%%%%%%%%%%%%%%%%%%%%%%%%%%%%%%%%%%%%%%%%%%%%%%%%%%%%%%%%%%%%%%%%%%%%%%%%%%%%%%%%%%%%%%%%%%%%%%%%%%%%%%%%%%%%%%%%%%%%%%%%%%%%%%%%%%%%%%%%%%%%%%%%%%%%%%%%%%%%%%%%%%%%%%%%%%%%%%%%%%%%%
\section{General Results}

In this section we present some new results related to nonseparating subsets contained in particular classes of groups generated by two elements. We begin with a lemma. This lemma is due to professor Walter Parry. 

\begin{lem}\label{lem:60} Let $A$ be a finite Abelian group generated by two elements such that $A/2A \cong\Z_{2}\oplus \Z_{2}$, Let $a$, $b$, $c$ and $d$ be elements of $A$ such that $a$ and $b$ have order 4, $2a=2b=d-c$ and the four sets $\{\pm a\}, \{\pm b\}, \{\pm c\}, \{\pm d\}$ are distinct. Then $H=\{\pm a,\ \pm b,\ \pm c ,\ \pm d \}$ is a nonseparating subset of $A$.\end{lem} 

\begin{proof} Let $e=2a$. Since $a$ has order 4, $e$ has order 2. Note that $2(a-b)=0$. Since $a\ne b$, $o(a-b)=2$. Also, note that $e\neq a-b$, otherwise $a=-b$ which is a contradiction. So $e$ and $a-b$ are distinct elements of $A$ of order 2.

Let $B$ be a cyclic subgroup of $A$ such that $A/B$ is cyclic. We first show that $|A/B|$ must be even. If $a+B$, $b+B$, $e+B$ were elements of order 1 in $A/B$, then $a-b$ and $e$ would be two distinct elements of order 2 contained in the cyclic group $B$ which is not possible. So the elements $a+B$, $b+B$, $e+B$ of the quotient group $A/B$ cannot all have order $1$. Since $o(a+B)|4$, $o(b + B)|4$, and $o(e+B)|2$, it follows that $|A/B|=2n$ for some $n\in\Z^{+}$.

Now let $C$ be the subgroup of order 2 in $A/B$. We analyze two cases.

Case 1: Suppose $e \in B$. Note that $2(a+B)=2(b+B)=e+B=B$. Then $a+B$ and $b+B$ are elements of $C$. Since $B$ is cyclic and $e$ is the unique element of order 2 in $B$, $a-b\notin B$. Hence $C=\{a+B,b+B\}$. On the other hand, it is clear that $c+B=c+e+B=d+B$. So the coset numbers for $H$ relative to $B$ and any generator of $A/B$ are $c_1=0$ and $c_4=n$ for the elements of $C$ and the value $c_{2}=c_{3}$ corresponding to the coset of $c+B$. This completes the first case.

Case 2: Suppose $e\notin B$. In this case $e+B$ is the element of order 2 in $A/B$. So both $a+B$ and $b+B$ are elements of order 4 in the cyclic group $A/B$, hence $a+B=\pm(b+B)$. We have that 4 divides the order of $A/B$, and so $n=2m$ for some integer $m$. It follows that both $a+B$ and $b+B$ have coset number $m$ independent of the generator of $A/B$. Furthermore, if $c+B$ has coset number $k$, then $-d+B=-c-e+B=(e+B)-(c+B)$ has coset number $n-k$. Since $k$ and $n-k$ are symmetric about $m$, we have $\{c_1,c_4\}=\{k,n - k\}$ and $c_2=c_3=m$. This completes the second case.\end{proof}

Based on the preceding lemma, Walter Parry found the following family of nonseparating subsets. Here we use his notation and terminology. Let $A=\Z_{m}\oplus \Z_{n}$, where $m$ and $n$ are even positive integers with $n|m$ and $m\geq4$. Lemma \ref{lem:60} leads to 3 choices for $a$, $b$, $c$ and $d$. Let $m$ and $n$ be even positive integers with $m$ divisible by $4$. Then we have the following 3 types of nonseparating sets:\\
\\
\textbf{Type 1:} $m \ne 4$, $n = 2, a = (\frac{m}{4}, 0), b = (\frac{m}{4}, 1), c = (1, 0), d = (\frac{m}{2} + 1, 0) \in \Z_{m}\oplus \Z_{2}$.\\
\\
\textbf{Type 2:} $n = 4$, $a=(0,1)$, $b=(\frac{m}{2}, 1)$, $c=(1,0)$, $d=(1, 2)\in\Z_{m}\oplus \Z_{4}$.\\
\\
\textbf{Type 3:} $m\ne8$ $n=2$, $a=(\frac{m}{4}, 0)$, $b=(\frac{m}{4},1)$, $c=(2, 0)$, $d=(\frac{m}{2}+2, 0)\in\Z_{m}\oplus \Z_{2}$.\\

In the next section we will describe our classification in terms of these types and some exceptional cases that do not arise from these types.\\

\newpage
Another family of nonseparating subsets can be found in Section 2 of \cite{KLS}. The authors point out the next result. Since no proof has been published yet, we provide a proof following the spirit of the proof of Lemma \ref{lem:60}.

\begin{lem}\label{lem:600} Let $A$ be a finite Abelian group generated by two elements such that $A/2A \cong\Z_{2}\oplus \Z_{2}$, Let $a$, $b$, $c$ and $d$ be elements of $A$ such that $a$ and $b$ have order 4, $2a\neq2b$, $c=a+b$, $d=a-b$ and the four sets $\{\pm a\}, \{\pm b\}, \{\pm c\}, \{\pm d\}$ are distinct. Then $H=\{\pm a,\ \pm b,\ \pm c ,\ \pm d \}$ is a nonseparating subset of $A$.\end{lem} 

\begin{proof} Let $B$ be a cyclic subgroup of $A$ such that $A/B$ is cyclic. We first show that $|A/B|$ must be even. Since $a$ and $b$ both have order of $4$, $o(2a+B)$ and $o(2b+B)$ are divisors $2$. If both $2a+B$ and $2b+B$ have order $1$, then $2a\in B$ and $2b\in B$, but that is impossible because $B$ is cyclic, $o(2a)=o(2b)=2$ and $2a\neq2b$. So at least one of $2a+B$, $2b+B$ has order of $2$. Thus, $|A/B|=2n$ for some $n\in\Z^{+}$.

Now let $C$ be the subgroup of order $2$ of $A/B$. Then $2a+B\in C$ and $2b+B\in C$ and $C=\{B,nz+B\}$ for any generator $z+B$ of $A/B$. Since we have shown that $2a+B$ and $2b+B$ cannot both be equal to $B$, that leads us to 3 cases:

Case 1: Suppose $2a+B=B$ and $2b+B=nz+B$. Since $o(2b+B)=2$ and $o(b)=4$, it follows that $o(b+B)=4$. So $|A/B|$ is divisible by $4$. Then $n=2m$ for some $m\in\Z^{+}$, and we without loss of generality we may assume that the coset number for $b+B$ is $m$. Since $A/B$ is cyclic and $2a+B=B$, either $a+B=B$ or $a+B=nz+B$. We analyze two subcases:
\begin{itemize}
\item If $a+B=B$, then the coset number for $a+B$ is $0$. Also, note that $c+B=a+b+B=b+B=b-a+B=-d+B$. Hence, regardless the choice of the generator of $A/B$, the coset numbers are $c_{1}=0$, $c_{2}=c_{3}=c_{4}=m$. 
\item If $a+B=nz+B$, then the coset number for $a+B$ is $n$. Also, note that $-c+B=-a+B-b+B=nz+B-mz+B=mz+B$ and that $d+B=a+B-b+B=nz+B-mz+B=mz+B$. Hence, regardless the choice of the generator of $A/B$, the coset numbers are $c_{1}=c_{2}=c_{3}=m$ and $c_{4}=n$. 
\end{itemize}

Case 2: Suppose $2a+B=nz+B$ and $2b+B=B$. The proof of this case is symmetric to the proof of the first case.

Case 3: Suppose $2a\notin B$ and $2b\notin B$. In this case, $2a+B=nz+B=2b+B$. Since $o(2a+B)=2$ and $o(a)=4$, it follows that $o(a+B)=4$. So $|A/B|=2m$ for some $m\in\Z^{+}$ and $a+B=\pm b+B$. Without loss of generality we may assume that the coset numbers for $a+B$ and $b+B$ are both equal to $m$. This yields
$c+B=a+b+B=2mz+B=nz+B$ and $d+B=a-b+B=B$. So, regardless the choice of the generator of $A/B$, the coset numbers are $c_{1}=0$, $c_{2}=c_{3}=m$ and $c_{4}=n$.\end{proof}

We now provide new properties of nonseparating subsets contained in a particular class of finite Abelian groups. Let $A=\Z_{2n}\oplus \Z_{2n}$, where $n=p_{1}^{k_{1}}p_{2}^{k_{2}}\cdots p_{r}^{k_{r}}, p_{i}$ prime, $p_{i}\geq3$, Assume that $H=\{\pm h_{1},\pm h_{2},\pm h_{3},\pm h_{4}\}$ is a nonseparating subset of $A$. Here $h_{i}=(x_{i},y_{i})$ where $x_{i},y_{i}\in \Z_{2n}$.

\begin{lem}\label{lem:61} Under the above settings $h_{i}\neq0$ for all $i$.\end{lem}

\begin{proof}  Draw a square grid with vertices at $(0,0)$, $(2n,0)$, $(0,2n)$ and $(2n,2n)$. We now proceed by contradiction. Suppose $H$ contains the point $(0,0)$. Without loss of generality we may assume that $h_{1}=(0,0)$. Choose $B$, a cyclic subgroup of $A$, so that it contains $h_{2}$ and $A/B$ is cyclic. By a group automorphism we may assume that $B=\langle (1,0)\rangle$. Then $c_{1}=c_{2}=0$ and so $c_{3}=0$. We may and do assume that $h_{3}\in B$ and $0<x_{2}<x_{3}\leq n$. Let $h_{4}=(x_{4},y_{4})$. Geometrically it follows that either $x_{4}=x_{2}$ or $x_{4}=2n-x_{2}$. 
 Assume that $x_{4}=x_{2}$. Let $B'=\langle(1,1)\rangle$. Note that $h_{4}\notin B'$; otherwise, $h_{2}$ or $h_{3}$ would be an element of $B'$, which is impossible. Now, let $a=(1,0)$ and consider the generator $a+B'$ of $A/B'$. The coset $0a+B'$ contains $h_{1}$ while the coset $x_{2}a+B'$ contains $h_{2}$, since $h_{3}\notin x_{2}a+B'$, $-h_{4}$ must be in the coset $x_{2}a+B'$. This forces $h_{4}=(x_{2},2x_{2})=x_{2}(1,2)$. So the cyclic subgroup $B''=\langle(1,2)\rangle$ contains $h_{1}$ and $h_{4}$. Then $h_{2}$ or $h_{3}$ must be in $B''$ which is impossible. Similarly, the case $x_{4}=2n-x_{2}$ lead us to a contradiction. \end{proof}

\begin{cor}\label{cor:61} Under the above settings, $H$ does not contain elements of order 2.\end{cor}
\begin{proof} Argue by contradiction. Without loss of generality suppose $y_{1}=0$. Translating $H$ by an element of order 2, we may assume that $h_{1}=(0,0)$. This contradicts the above lemma.\end{proof}

\begin{lem}\label{lem:62} Let be $A$ and $n$ be as above where $n$ is not a multiple of $3$. If $B$ be a cyclic subgroup of $A$ such that $A/B$ is cyclic, then $B$ contains at most one of the elements $h_{1}$, $h_{2}$, $h_{3}$, $h_{4}$.\end{lem}

\begin{proof}  Draw a square grid with vertices at $(0,0)$, $(2n,0)$, $(0,2n)$ and $(2n,2n)$ and proceed by contradiction. Suppose there is a cyclic subgroup $B$ such that $A/B$ is cyclic and $h_{1},h_{2}\in B$. Since $H$ is nonseparating, $h_{3}\in B$. By a group automorphism we may assume that $B=\langle (1,0)\rangle$. Since $H$ has no elements of order 2, we may assume that $h_{1}=(x_{1},0)$, $h_{2}=(x_{2},0)$, $h_{3}=(x_{3},0)$ with $0<x_{1}<x_{2}<x_{3}<n$. Geometrically, either $x_{4}=x_{2}$ or $x_{4}=2n-x_{2}$. Assume $x_{4}=x_{2}$. We know that cyclic groups do not contain nonseparating subsets, so $0<y_{4}<2n$. Let $B'=\langle(1,1)\rangle$ and consider the generator $a+B'$ with $a=(1,0)$. Because $H$ is nonseparating, $h_{4}\notin B'$; otherwise, $c'_{1}=0$, $c'_{2}=x_{1}$ and $c'_{3}=x_{2}$. Since $x_{1}$, $x_{2}$ and $x_{3}$ are distinct coset numbers, $-h_{4}$ must be in the coset $x_{2}a+B'$. This forces $-h_{4}=(2n-x_{2},2n-2x_{2})$ and so $h_{4}=(x_{2},2x_{2})$. Now let $B''=\langle(-1,1)\rangle$ and consider the generator $a+B''$ with $a=(-1,0)$. Because $x_{1}$, $x_{2}$ and $x_{3}$ are distinct coset numbers we must have $c''_{2}=c''_{3}=x_{2}$. It follows that $h_{4}\in x_{2}a+B$. This forces $h_{4}=(x_{2},2n-x_{2})$. Then $2n=3x_{2}$ and so $3|n$. This contradicts our main assumption on $n$. Similarly, the case $x_{4}=2n-x_{2}$ lead us to a contradiction.\end{proof}

\begin{cor} Let $A'$ be a proper subgroup of $A$ generated by two elements. Suppose that $H=\{\pm h_{1},\pm h_{2},\pm h_{3},\pm h_{4}\}$ is a nonseparating subset of $A$ contained in $A'$. If $B'$ is a cyclic subgroup of $A'$ such that $A'/B'$ is cyclic, then $B'$ contains at most one of the elements $h_{1}$, $h_{2}$, $h_{3}$, $h_{4}$.\end{cor}
\begin{proof}  There exists a cyclic subgroup $B$ of $A$ such that $A/B$ is cyclic and $A'\cap B=B'$. By Lemma \ref{lem:62}, $B$ contains at most one of the elements $h_{1}$, $h_{2}$, $h_{3}$, $h_{4}$. Since $A'\cap B=B'$, $B'$ contains at most one of the elements $h_{1}$, $h_{2}$, $h_{3}$, $h_{4}$.\end{proof}

%%%%%%%%%%%%%%%%%%%%%%%%%%%%%%%%%%%%%%%%%%%%%%%%%%%%%%%%%%%%%%%%%%%%%%%%%%%%%%%%%%%%%%%%%%%%%%%%%%%%%%%%%%%%%%%%%%%%%%%%%%%%%%%%%%%%%%%%%%%%%%%%%%%%%%%%
\section{Classifying Nonseparating Subsets}
The first author wrote a computer program to find all the nonseparating subsets of finite Abelian group $A$ generated by 2 elements of the form $A=\Z_{m}\oplus \Z_{n}$ where $n, m$ are even positive integers with $n|m$ and $m\geq4$. The program outputs every nonseparating subset for $m\leq 100$. Using the data obtained by the program, we classify all nonseparating subsets contained in subgroups of exponent at most 10.\\
\\
\textbf{Classifying nonseparating subsets in $A=\Z_{2}\oplus\Z_{2}$.}
\begin{lem}\label{lem:100} The group $A=\Z_{2}\oplus \Z_{2}$ does not contain any nonseparating subset.
\end{lem} 
\begin{proof} For any cyclic subgroup $B$ of $A$ for which $A/B$ is cyclic it follows that $c_{1}=c_{2}=0$ and $c_{3}=c_{4}=1$.
\end{proof}\
\\
\textbf{Classifying nonseparating subsets in $A=\Z_{4}\oplus \Z_{2}$.}
\begin{lem}\label{lem:101} There does exist only one Hurwitz class of nonseparating subsets for $A=\Z_{4}\oplus \Z_{2}$. A representative for this class is $$H=\{(0,0), \pm(1,0), \pm(2,0), \pm(1,1)\}.$$\end{lem}
\begin{proof} To show that $\{(0,0), \pm(1,0), \pm(2,0), \pm(1,1)\}$ is a nonseparating subset of $A$ it suffices to apply Lemma \ref{lem:60} to $a=(1,0)$, $b=(2,0)$, $d=(2,0)$, $c=(0,0)$. We now show that there is only one Hurwitz class. Group inversion generates an equivalence relation on $A$ whose equivalence classes are 
$$\{(0,0)\}\hspace{.2in}\{(2,0)\}\hspace{.2in}\{(0,1)\}\hspace{.2in}\{(2,1)\}\hspace{.2in}\{\pm(1,0)\}\hspace{.2in}\{\pm(1,1)\}$$
Let $H$ be a nonseparating subset of $A$. From Lemma \ref{lem:100} and Lemma \ref{lem:a05} it follows that $H$ must contain at least one element of order 4. If $H$ contains no elements of order 2, then $H\subseteq\{(0,0),\pm(1,0), \pm(1,1)\}$ which lead us to a contradiction. So $H$ must contain at least one element of order 2, say $\tau$. By Lemma \ref{lem:55}, $H'=H+\tau$ is a nonseparating subset of $A$. It follows that $(0,0)\in H'$. Also, note that $H'$ must contain at least one element of order 4; otherwise $H'$ would be the subgroup $\langle 2\rangle\oplus\Z_{2}\cong\Z_{2}\oplus\Z_{2}$ which contradicts Lemma \ref{lem:100}. By a group automorphism of $A$ we may assume that $(1,0)\in H'$. So without loss of generality we may assume that $(0,0)\in H$ and that $(1,0)\in H$. Let $B=\langle(1,0)\rangle$, it is clear that $A/B$ is cyclic. Since $c_2 = c_3$, it follows that $(2,0)\in H'$. Now let $B'=\langle(0,1)\rangle$ and consider the generator of $A/B'$ as $(1,0)+B'$. Since $c'_{2}=c'_{3}$, it follows that $(1,1)\in H$. This completes the proof.\end{proof}\
\\
\textbf{Classifying nonseparating subsets in $A=\Z_{4}\oplus\Z_{4}$.}
\begin{lem}\label{lem:102}  There are exactly three Hurwitz classes of nonseparating subsets for $A=\Z_{4}\oplus \Z_{4}$. The following subsets are representatives for each Hurwitz class:
$$H_{1}=\{(0,0);\pm(1,0);\pm(1,2);(2,0)\},\ H_{2}=\{\pm(1,0),\pm(0,1),\pm(2,1),\pm(1,2)\},$$
$$H_{3}=\{\pm(1,0),\pm(0,1),\pm(1,1),\pm(3,1)\}.$$
\end{lem}
\begin{proof} Note that $H_{1}\subseteq \Z_{4}\oplus\langle2\rangle\cong\Z_{4}\oplus\Z_{2}$ and that $H_{1}$ is a nonseparating subset of $\Z_{4}\oplus\langle2\rangle$. By Lemma \ref{lem:56} it follows that $H_{1}$ is a nonseparating subset of $A$. To show that $H_{2}$ is a nonseparating subset of $A$ it suffices to apply Lemma \ref{lem:60} to $a=(1,2)$, $b=(1,0)$, $d=(2,1)$, $c=(0,1)$. To show that $H_{3}$ is a nonseparating subset of $A$ it suffices to apply Lemma \ref{lem:600} to $a=(1,0)$, $b=(0,1)$, $d=(3,1)$, $c=(1,1)$. Moreover, $H_{1}$ is a nonseparating subset of type 3 while $H_{2}$ is a nonseparating subset of type 2.\\

We now show that there are exactly three Hurwitz classes of nonseparating subsets. Group inversion generates an equivalence relation on $A$ whose equivalence classes are 
$$\{(0,0)\}\hspace{.2in}\{(2,0)\}\hspace{.2in}\{\pm(1,0)\}\hspace{.2in}\{\pm(1,1)\}\hspace{.2in}\{\pm(1,2)\}$$
$$\{(0,2)\}\hspace{.2in}\{(2,2)\}\hspace{.2in}\{\pm(0,1)\}\hspace{.2in}\{\pm(2,1)\}\hspace{.2in}\{\pm(1,3)\}$$
Let $H$ be a nonseparating subset of $A$. The subset $H$ must contain at least one element of order 4; otherwise $H$ would be the subset $\langle 2\rangle\oplus\langle 2\rangle\cong\Z_{2}\oplus\Z_{2}$. We first show that if $H$ contains at least one element of order 2, then $H$ is in the equivalence class of $H_{1}=\{(0,0);\pm(1,0);\pm(2,0);(1,2)\}$. Let $b$ an element of order 2 in $H$, then $H'=H+b$ is a nonseparating set containing $(0,0)$ and an element of order 4. By a group automorphism on $A$, we may assume that $H'$ contains $\pm(1,0)$. So without loss of generality we may assume that $H$ contains $(0,0)$ and $\pm(1,0)$. Let $B=\langle(1,0)\rangle$, one verifies that $(2,0)\in H$. Now take $B'=\langle(0,1)\rangle$ and consider the generator $(1,0)+B'$ of $A/B'$. Then exactly one of the following must be contained in $H$: 
$\{\pm(1,1)\},\ \{\pm(1,2)\},\ \{\pm(1,3)\}$. Now let $B''=\langle(1,2)\rangle$. Since $H$ is a nonseparating subset, any generator for $A/B''$ lead us to $c_{1}=c_{2}=c_{3}=0$. Thus, $\pm(1,2)\in H$.\\

We show that if $H$ contains no elements of order 2, then $H$ is either in the Hurwitz class of
$H_{2}$ or in the Hurwitz class of $H_{3}$. First of all, notice that $H$ does not contain $(0,0)$; otherwise, it would be in the equivalence class of $H_{1}$. So $H$ contains exactly four of the following sets
$$\{\pm(1,0)\}\hspace{.2in}\{\pm(0,1)\}\hspace{.2in}\{\pm(1,1)\}\hspace{.2in}\{\pm(1,2)\}\hspace{.2in}\{\pm(2,1)\}\hspace{.2in}\{\pm(1,3)\}$$
Then $H$ contains a basis of $A$. By a group automorphism of $A$, we may assume that $H$ contains $\pm(1,0)$ and $\pm(0,1)$, so $H$ is in the Hurwitz class of at least one of the following subsets
$$J_{1}=\{\pm(1,0),\pm(0,1),\pm(1,2),\pm(2,1)\}\hspace{.4in}J_{2}=\{\pm(1,0),\pm(0,1),\pm(1,2),\pm(1,1)\}$$
$$J_{3}=\{\pm(1,0),\pm(0,1),\pm(1,2),\pm(3,1)\}\hspace{.4in}J_{4}=\{\pm(1,0),\pm(0,1),\pm(2,1),\pm(1,1)\}$$
$$J_{5}=\{\pm(1,0),\pm(0,1),\pm(2,1),\pm(3,1)\}\hspace{.4in}J_{6}=\{\pm(1,0),\pm(0,1),\pm(1,1),\pm(3,1)\}.$$\\
In this paragraph we reduce the number of representatives. The automorphism $\varphi(x,y)=(y,x)$ takes $J_{2}$ into $J_{4}$ and the affine mapping $\phi(x,y)=(x+y,y)+(2,0)$ takes 
$J_{4}$ into $J_{6}$. The automorphism $f(x,y)=(y,x)$ takes $J_{3}$ into $J_{5}$ and $J_{5}+(2,0)=J_{4}$. This reduces to two representatives: $J_{1}$ and $J_{6}$. It remains to show that  $J_{1}$ and $J_{6}$ are not in the same Hurwitz class. To do so, let $L =(a_{ij})$ be a $2\times 2$ matrix with integer coefficients that induces an automorphism on $A$ and let $b=(b_{1},b_{2})\in A$ such that $2b=0$. Then, 
$$L\left[{\begin{array}{*{20}c}
    1\vspace{.05in}\\
    0 
 \end{array}}\right]+\left[{\begin{array}{*{20}c}
    b_{1}\vspace{.05in}\\
    b_{2} 
 \end{array}}\right]\equiv\left[{\begin{array}{*{20}c}
    m_{11} \vspace{.05in}\\
    m_{21}
 \end{array}}\right](\text{mod}\ 2)\hspace{.4in}L\left[{\begin{array}{*{20}c}
    0\vspace{.05in}\\
    1 
 \end{array}}\right]+\left[{\begin{array}{*{20}c}
    b_{1}\vspace{.05in}\\
    b_{2} 
 \end{array}}\right]\equiv\left[{\begin{array}{*{20}c}
    m_{12} \vspace{.05in}\\
    m_{22}
 \end{array}}\right](\text{mod}\ 2)$$
 
 $$L\left[{\begin{array}{*{20}c}
    2\vspace{.05in}\\
    1 
 \end{array}}\right]+\left[{\begin{array}{*{20}c}
    b_{1}\vspace{.05in}\\
    b_{2} 
 \end{array}}\right]\equiv\left[{\begin{array}{*{20}c}
    m_{12} \vspace{.05in}\\
    m_{22}
 \end{array}}\right](\text{mod}\ 2)\hspace{.4in}
 L\left[{\begin{array}{*{20}c}
    1\vspace{.05in}\\
    2 
 \end{array}}\right]+\left[{\begin{array}{*{20}c}
    b_{1}\vspace{.05in}\\
    b_{2} 
 \end{array}}\right]\equiv\left[{\begin{array}{*{20}c}
    m_{11} \vspace{.05in}\\
    m_{21}
 \end{array}}\right](\text{mod}\ 2)$$\\
On the other hand, the elements of $J_{6}$ verifies the following
$$\left[{\begin{array}{*{20}c}
    1\vspace{.05in}\\
    0 
 \end{array}}\right]\equiv\left[{\begin{array}{*{20}c}
    1 \vspace{.05in}\\
    0
 \end{array}}\right](\text{mod}\ 2)\hspace{1in}
 \left[{\begin{array}{*{20}c}
    0\vspace{.05in}\\
    1 
 \end{array}}\right]\equiv\left[{\begin{array}{*{20}c}
    0 \vspace{.05in}\\
    1
 \end{array}}\right](\text{mod}\ 2)$$
 
 $$\left[{\begin{array}{*{20}c}
    1\vspace{.05in}\\
    1 
 \end{array}}\right]\equiv\left[{\begin{array}{*{20}c}
    1 \vspace{.05in}\\
    1
 \end{array}}\right](\text{mod}\ 2)\hspace{1in} 
 \left[{\begin{array}{*{20}c}
    3\vspace{.05in}\\
    1 
 \end{array}}\right]\equiv\left[{\begin{array}{*{20}c}
    1 \vspace{.05in}\\
    1
 \end{array}}\right](\text{mod}\ 2)
 $$
Since $\{(1,0);(0,1);(1,1)\}$ and $\{(m_{11},m_{21});(m_{21},m_{22})\}$ have different cardinalities, we conclude that $J_{6}$ cannot be an element of the Hurwitz class of $J_{1}$. This completes the proof. The computer program also verifies the result.\end{proof}\
\\
\textbf{Classifying nonseparating Sets in $A=\Z_{6}\oplus\Z_{6}$.}\vspace{.05in}\\
We use the isomorphism $T:\Z_{6}\oplus \Z_{6}\to\Z_{2}\oplus\Z_{2}\oplus \Z_{3}\oplus\Z_{3}$ defined by $T(x,y)=(x,y,x,y)$ to identify $A=\Z_{6}\oplus \Z_{6}$ with the group $\Z_{2}\oplus\Z_{2}\oplus \Z_{3}\oplus \Z_{3}$. By abuse of notation we denote $\Z_{2}\oplus \Z_{2}\oplus \Z_{3}\oplus \Z_{3}$ by $A$ and assume that $H=\{\pm h_{1},\pm h_{2},\pm h_{3},\pm h_{4}\}$ is a nonseparating subset of $A$. Here $h_{i}=(a_{i},b_{i})$ where $a_{i}\in\Z_{2}\oplus \Z_{2}$ and $b_{i}\in\Z_{3}\oplus\Z_{3}$. Corollary \ref{cor:61} tells us that any nonseparating subset of $A$  contains no elements of order 2. So $\{\pm b_{1},\pm b_{2},\pm b_{3},\pm b_{4}\}$ is a subset of $(\Z_{3}\oplus \Z_{3})\setminus\{(0,0)\}$.\\
\begin{thm} $\{\pm b_{1},\pm b_{2},\pm b_{3},\pm b_{4}\}=(\Z_{3}\oplus \Z_{3})\setminus\{(0,0)\}$.\end{thm}
\begin{proof} Argue by contradiction. Suppose $b_{1}=b_{2}$. Translating $H$ by an element of order 2, we may assume that $h_{1}=(0,b_{1})$. Since $b_{1}=b_{2}$ and $a_{1}=0$, there exists a cyclic subgroup $B=\langle \alpha\rangle\oplus\langle b_{1}\rangle$ of order 6 containing $h_{1}$ and $h_{2}$. Then $c_{1}=c_{2}=c_{3}=0$ and so $h_{3}\in B$ or $h_{4}\in B$. Without loss of generality $h_{3}\in B$, whence $h_{3}=(a_{3},b_{1})$. Since $h_{1}\neq\pm h_{3}$ it follows that $a_{3}\neq0$. So the subgroup $B$ is generated by $h_{3}=(a_{3},b_{1})$. This forces $h_{2}=(0,-b_{1})$ or $h_{2}=(a_{3},0)$. Since $h_{2}=(a_{2},b_{1})$ and $o(b_{1})=3$ we finally reach a contradiction.\end{proof}\
\\
As an immediate consequence any nonseparating set of $\Z_{2}\oplus \Z_{2}\oplus \Z_{3}\oplus \Z_{3}$ has the form $H=\{\pm(a_{1},(1,0)),\pm(a_{2},(0,1)),\pm(a_{3},(2,1)),\pm(a_{4},(2,2))\}$.
Since the subset $(\Z_{3}\oplus \Z_{3})\setminus\{(0,0)\}$ is invariant under automorphisms of $\Z_{3}\oplus \Z_{3}$, the Hurwitz class of a nonseparating subset $H$ of $A$ depends only on the choice of the elements $a_{1}, a_{2}, a_{3}, a_{4}$. We now classify the nonseparating subsets of  $A=\Z_{6}\oplus\Z_{6}$.\\
\\
\textbf{Case 1} All the elements of the set $\{a_{1},a_{2},a_{3},a_{4}\}$ are the same. Translating by an element of order two we obtain the set
$$H_{1}=\{\pm(0,0,1,0),\pm(0,0,0,1),\pm(0,0,2,1),\pm(0,0,2,2)\}.$$
Note that $T^{-1}(H_{1})=\{\pm(0,2),\pm(2,0),\pm(2,4),\pm(2,2)\}$, which is actually a non-separating subset of $A=\Z_{6}\oplus\Z_{6}$ (see Example 10.4 of \cite{CFPP}).\\ 
\\
\textbf{Case 2} All the elements of the set $\{a_{1},a_{2},a_{3},a_{4}\}$ are distinct. Without loss of generality, we may assume that $a_{1}=(1,0)$, $a_{2}=(0,1)$, $a_{3}=(1,1)$ and $a_{4}=(0,0)$. So we obtain 
$$H_{2}=\{\pm(1,0,1,0),\pm(0,1,0,1),\pm(1,1,2,1),\pm(0,0,2,2)\}.$$
Note that $T^{-1}(H_{2})=\{\pm(1,0),\pm(0,1),\pm(5,1),\pm(2,2)\}$, which is actually a non-separating subset of $A$. This was verified with the assistance of the computer program (see Appendix A).\\ 
\\
\textbf{Case 3} Two pair of elements of $\{a_{1},a_{2},a_{3},a_{4}\}$ are the same but not all the elements are the same. Without loss of generality, we may assume that $a_{1}=a_{3}=(1,0)$ and $a_{2}=a_{4}=(0,1)$. So we obtain
$$H_{3}=\{\pm(1,0,1,0),\pm(0,1,0,1),\pm(1,0,2,1),\pm(0,1,2,2)\}.$$
Note that $T^{-1}(H_{3})=\{\pm(1,0),\pm(0,1),\pm(5,4),\pm(2,5)\}$, which is actually a non-separating subset of $A$. This was verified with the assistance of the computer program (see Appendix A).\\ 
\\
\textbf{Case 4}  Exactly two elements of $\{a_{1},a_{2},a_{3},a_{4}\}$ are the same and the other two are distinct. Without loss of generality, we may assume that $a_{1}=a_{2}=0$, $a_{3}=(1,0)$ and $a_{4}=(0,1)$. So we obtain
$$H_{4}=\{\pm(0,0,1,0),\pm(0,0,0,1),\pm(1,0,2,1),\pm(0,1,2,2)\}.$$
Note that $T^{-1}(H_{4})=\{\pm (2,0);\pm(0,2);\pm(5,4);\pm(2,5)\}$. It is not difficult to show that $T^{-1}(H_{4})$ is not a nonseparating subset. To do so, consider the cyclic subgroup $B=\langle(0,1)\rangle$ of $\Z_{6}\oplus \Z_{6}$ and the generator $(1,0)+B$ of $A/B$. It follows that $c_{1}=0$, $c_{2}=1$, $c_{3}=c_{4}=2$. So $H_{4}$ is not a nonseparating subset. Therefore, there does not exist a nonseparating subset in this case.\\
\\
\textbf{Case 5} Exactly three elements of $\{a_{1},a_{2},a_{3},a_{4}\}$ are the same. Without loss of generality, we may assume that $a_{1}=a_{2}=a_{3}=0$ and $a_{4}=(1,0)$. So we obtain
$$H_{5}=\{\pm(0,0,1,0),\pm(0,0,0,1),\pm(0,0,2,1),\pm(1,0,2,2)\}.$$
Note that $T^{-1}(H_{5})=\{\pm (2,0);\pm(0,2);\pm(4,2);\pm(1,4)\}$. It is not difficult to show that $T^{-1}(H_{5})$ is not a nonseparating subset. To do so, consider the cyclic subgroup $B=\langle(0,1)\rangle$ of $\Z_{6}\oplus \Z_{6}$ and the generator $(1,0)+B$ of $A/B$. It follows that $c_{1}=0$, $c_{2}=1$, $c_{3}=c_{4}=2$. So $H_{5}$ is not a nonseparating subset. Therefore, there does not exist a nonseparating subset in this case.\\
\\
\textbf{Classifying nonseparating subsets in $A=\Z_{8}\oplus \Z_{2}$.}\vspace{.05in}\\
With the assistance of the computer program we obtained the following information. Plus/minus are omitted in the table below.
\begin{center}
\begin{tabular}{|c|c|c|}
\hline
Hurwitz class representative&number of elements in the Hurwitz class\\
\hline
$\{(0,0),(2,0),(4,0),(2,1)\}$&$2$\\
\hline
$\{(1,0),(2,0),(3,0),(2,1)\}$&$2$\\
\hline
\end{tabular}
\end{center}\
\\
Note that the subset $H_{1}=\{(0,0),\pm(2,0),(4,0),\pm(2,1)\}$ is a nonseparating subset of the subgroup $A'=\langle2\rangle\oplus\Z_{2}$ which is isomorphic to $\Z_{4}\oplus\Z_{2}$. On the other hand, $H_{2}=\{\pm(0,0),\pm(2,0),\pm(2,1), \pm(4,0)\}$ is a nonseparating subset of Type 1.\\
\\
\textbf{Classifying nonseparating subsets in $A=\Z_{8}\oplus\Z_{4}$.}\vspace{.05in}\\
In this case, the order of the group $A$ is 32 and there are several subgroups $B$ for which $A/B$ is cyclic, so it is convenient to use the computer program designed by the first author. The program outputs one representative of for each Hurtwiz class. Almost all of these representatives correspond to either type 1, type 2 or type 3; however, there is one special subset that does not correspond to any of these types, we call this nonseparating subset exceptional. We provide an alternative proof for this exceptional case. There are exactly 41 nonseparating subsets in $\Z_{8}\oplus\Z_{4}$ and $9$ distinct Hurwitz classes. Plus/minus are omitted in the table below.\\
%For all the nonseparating susbsets in $\Z_{4}\times \Z_{8}$, my program gives the following results, by \ref{lem:56}, many of the nonseparating sets are coming from subgroups of $\Z_{4}\times \Z_{8}$:\\
%\begin{lem}\label{lem:104} $H$ = $\{\pm(1,0), \pm(2,0), \pm(3,0), \pm(2,1)\}$ is the only nonseparating set in $\Z_{8}\times \Z_{2}$ and it's a TYPE 1 nonseparating set.\end{lem}
%\begin{proof} In $\Z_{8}\times \Z_{2}$, $(0,0)$ has order 1, $(4,0)$, $(4,1)$ and $(0,1)$ has order 2, $\pm(2,0), \pm(2,1)$ has order 4, rest of the elements have order 8. Let's assume $H$ is a nonseparating set in  $\Z_{4}\times \Z_{2}$, If $H$ contain an element of order 2, then by translating by an element of order 2 and \ref{lem:55}, $(0,0)$ is contained in $H$. By \ref{lem:a06}, $H$ must contains element of order 4. Then again by translating by an element of order 2 and \ref{lem:55}, without losing generality we can say $(0,0)$ and $(2,0)$ are contained in $H$. Since $c_2 = c_3$, let $B\langle(0,1)\rangle$ then either$(2,1)$ in $H$, or $(1,1)$ and $(1,0)$ in $H$ and together form a nonseparating set, but consider $B=\langle(0,1)\rangle$, and generator $a=(3,0)$, then the second case leads to $c_2 = 2$ and $c_3 = 3$, therefore $(2,1)$ in $H$. Let $B=\langle(1,0)\rangle$, then either $(4,0)$ in $H$ or $(3,0)$ in $H$. For the second case again we consider $B=\langle(0,1)\rangle$, and generator $a=(3,0)$, then the second case leads to $c_2 = 1 and c_3 = 2$, Therefore up to automorphism and translation by an element of order 2, if $H$ contain an element of order 2 then $H$ = $\{\pm(0,0), \pm(2,0), \pm(4,0), \pm(2,1)\}$ is the only nonseparating set in $\Z_{8}\times \Z_{2}$ and it's a TYPE 3 nonseparating set coming from $\Z_{4}\times \Z_{2}$ by \ref{lem:56}. If $H$ does not contain an element of order 2, then consider $B$ = $\langle(1,0)\rangle$, either $(1,0)$ and $(3,0)$ in $H$ or $(1,1), (2,1) and (3,1)$ all in $H$, by translating by $(0,1)$, both cases actually equals to each other, and if we consider $B=\langle(1,0)\rangle$, for $c_2 = c_3$,  $H$ = $\{\pm(1,0), \pm(2,0), \pm(3,0), \pm(2,1)\}$ is the only nonseparating set in $\Z_{8}\times \Z_{2}$ and this completes the proof. My program verifies with the result.\end{proof}

\begin{center}
\begin{tabular}{|c|c|c|}
\hline
Hurwitz class representative& cardinality of Hurwitz class & Type\\
\hline
$\{(0,0),(2,0),(4,0),(2,2)\}$&$2$&$3$\\
\hline
$\{(0,0),(0,1),(4,1),(0,2)\}$&$4$&$3$\\
\hline
$\{(1,0),(2,0),(3,0),(2,2)\}$&$4$&$1$\\
\hline
$\{(1,1),(2,1),(3,1),(6,1)\}$&$8$&$2$\\
\hline
$\{(1,0),(1,1),(1,2),(3,1)\}$&$8$& Exceptional\\
\hline
$\{(2,0),(0,1),(2,1),(4,1)\}$&$8$& Lemma 3.2\\
\hline
$\{(2,0),(0,1),(6,1),(2,2)\}$&$4$& Lemma 3.2 \\
\hline
$\{(2,0),(0,1),(4,1),(2,2)\}$&$2$&$2$\\
\hline
$\{(0,1),(2,1),(4,1),(6,1)\}$&$1$&$2$\\
\hline
\end{tabular}
\end{center}\
\\
%$\{(0,0)(0,2)(0,4)(2,2)\}$ is the root, the number of nonseparating sets in this equivalence class is: 2, this is the copy coming from $\Z_{2}\times \Z_{4}$
%$\{(0,0)(1,0)(1,4)(2,0)\}$ is the root, the number of nonseparating sets in this equivalence class is: 4, this is the copy coming from $\Z_{2}\times \langle(2)\rangle$ which is $\cong$ $\Z_{2}\times \Z_{4}$
%$\{(0,2)(1,1)(1,5)(2,2)\}$ is the root, the number of nonseparating sets in this equivalence class is: 4, this is the copy coming from $\Z_{2}\times \Z_{8}$
%$\{(1,1)(1,2)(1,3)(1,6)\}$ is the root, the number of nonseparating sets in this equivalence class is: 8, this is a new TYPE 2 nonseparating set
%$\{(0,1)(1,3)(1,5)(2,3)\}$ is the root, the number of nonseparating sets in this equivalence class is: 8, this is a new exceptional nonseparating set
%$\{(0,2)(1,0)(1,2)(1,4)\}$ is the root, the number of nonseparating sets in this equivalence class is: 8, this is the copy coming from $\Z_{4}\times \Z_{4}$
%$\{(0,2)(1,0)(1,6)(2,2)\}$ is the root, the number of nonseparating sets in this equivalence class is: 4
%$\{(0,2)(1,0)(1,4)(2,2)\}$ is the root, the number of nonseparating sets in this equivalence class is: 2, this is the copy coming from $\Z_{4}\times \Z_{4}$
%$\{(1,0)(1,2)(1,4)(1,6)\}$ is the root, the number of nonseparating sets in this equivalence class is: 1\\
We now provide a topological-algebraic proof of the nonseparability the exceptional subset $\{\pm(1,0);\pm(1,1);\pm(1,2);\pm(3,1)\}$. We begin with a basic claim.

\begin{clm}\label{clm:1}
If $B$ is a cyclic subgroup of $A=\Z_{8}\oplus\Z_{4}$ such that $|B|=4$ and $A/B$ is cyclic, then $(4,0)\notin B$.
\end{clm}
\begin{proof}
Let $z+B$ be a generator of $A/B$. It follows that the order of $z$ is a multiple of $8$. Since the maximum order of an element in $A$ is $8$, the order of $z=(a,b)$ must be $8$. This forces $a$ to be odd. Note that
$4z=4(a,b)=(4a,0)=(4,0)$. If $(4,0)$ were an element of $B$, we would have $4(z+B)=B$. So $z+B$ would not be a  generator of $A/B$ which lead us to a contradiction. Therefore $(4,0)\notin B$.
\end{proof}

\begin{thm} The set $H=\{\pm(1,0);\pm(1,1);\pm(1,2);\pm(3,1)\}$ is a nonseparating subset of the group $A=\Z_{8}\oplus\Z_{4}$.
\end{thm}
\begin{proof} Consider the lattices $\Lambda_{2}=\Z^{2}$ and $\Lambda_{1}=\langle(4,0),(0,2)\rangle$. Let $\Phi:\R^{2}\to\R^2$ be the linear mapping defined by $\Phi(x,y)=(4x,2y)$. It is clear that $\Phi(\Lambda_{2})=\Lambda_{1}$
so the mapping $\Phi$ induces a NET map $g:\R^{2}/\Gamma_{1}\to\R^{2}/\Gamma_{1}$, where $\Gamma_{1}$ is the group of isometries on $\R^2$ generated by rotations $180^{\circ}$ about elements of $\Lambda_{1}$. We identify the quotient space $\R^{2}/\Gamma_{1}$ with $S^2$. Bracket notation indicates a point in $S^2$; i.e., $[u,v]$ represents a point in $S^{2}$. It is not difficult to see the critical values of $g$ are $\a=[0,0]$, $\b=[4,0]$, $\c=[4,2]$ and $\d=[0,2]$. It is furthermore true that each critical point of $g$ is a simple. Below is part of the critical portrait of $g$.
\[\xymatrix{
[1,2],[1,0],[3,0],[3,2]\ar[r]^-{2,2,2,2} &\b \ \ar[rd]\\
[1,1],[3,1],[5,1],[7,1]\ar[r]^-{2,2,2,2} &\c \ar[r] & \a\ar@(dr,ur)[]\\
[0,1],[2,1],[4,1],[6,1]\ar[r]^-{2,2,2,2} &\d\ar[ru]\\ 
}\]
Note that $g([2,0])=g([2,2])=\a$ and that $[2,0], [2,2]$ are critical points of $g$. Also, note that $g=g_{1}\circ g_{2}$ where $g_{1}:S^{2}\to S^{2}$ and $g_{2}:S^{2}\to S^{2}$ are the branched covering mappings induced by the linear mappings $L_{1}(x,y)=(2x,2y)$ and $L_{2}(x,y)=(2x,y)$ respectively. Since $P_{g}=\{\a,\b,\c,\d\}$ and $g_{1}$ induce the identity mapping on Teichm\"uller space, we can easily use a core arc argument. Let $\gamma$ be an arc in connecting the points $\a$ and $\c$ and suppose that $\gamma\setminus\{\a,\c\}\subset S^{2}\setminus P_{g}$. The preimage of $\gamma$ under $g$ has two connected components say $\gamma_{1}$, $\gamma_{2}$. Exactly one of these components contains the point $[1,1]$, the other component contains the point $[3,1]$ while the points $[1,0]$ and $[1,2]$ are not contained in the disjoint union of $\gamma_{1}, \gamma_{2}$. Similarly, let $\delta$ be an arc  connecting the points $\a$ and $\b$ and suppose that $\delta\setminus\{\a,\b\}\subset S^{2}\setminus P_{g}$. The preimage of $\delta$ under $g$ has two connected components say $\delta_{1}$, $\delta_{2}$. Exactly one of these components contains the point $[1,2]$, the other component contains the point $[1,0]$ while the points $[1,1]$ and $[1,3]$ are not contained in the disjoint union of $\delta_{1}, \delta_{2}$. Note that $\deg(g:\gamma_{i}\to\gamma)=\deg(g:\delta_{i}\to\delta)=4$. By Theorem 4.1 of \cite{CFPP} it follows that $c_{2}=c_{3}$ for any cyclic subgroup $B$ of $A$ with $|B|=8$ such that $A/B$ is cyclic.

Now let $B$ be a cyclic subgroup of $A$ with $|B|=4$ such that $A/B$ is cyclic. We show that $c_2=c_3$ regardless the choice of the generator of $A/B$. Let $h_1=(1,0)$, $h_2=(1,1)$, $h_3=(1,2)$, and $h_4=(3,1)$. It is clear that $o(h_{i}+B)=8$ for all $i$. So given any generator $z+B$ of $A/B$, we have $\pm h_i+B\in\{z+B, 3z+B\}$ for all $i$. We now go by cases.\\
%$$\{h_{1}+B,h_{2}+B,h_{3}+B,h_{4}+B\}\subseteq\{z+B,3z+B,5z+B,7z+B\}.$$ 
Case 1: Suppose  $(1,1)+B=(3,1)+B$. Then $(2,0)\in B$ and so $(4,0)\in B$. This contradicts claim \ref{clm:1}.\\
Case 2: Suppose $(1,1)+B=-(3,1)+B$. Then $(4,2)\in B$. If $(1,2)+B=(1,0)+B$, then $(0,2)\in B$. Hence $(4,0)\in B$ which contradicts claim \ref{clm:1}. If $(1,2)+B=-(1,0)+B$, then $(2,2)\in B$ and again $(4,0)\in B$ which contradicts claim \ref{clm:1}. So in this case, $(1,2)+B\neq (1,0)+B$ and $(1,2)+B\neq-(1,0)+B$. Thus $c_{2}=c_{3}$.\\
%Then $(6,0)\in B$ and so $(4,0)\in B$. This contradicts claim \ref{clm:1}.\\
Case 3: Suppose $(1,1)+B=(1,0)+B$. Then $(0,1)\in B$, so $(1,2)+B=(1,1)+B$. This shows that $(1,1)$,$(1,0)$, and $(1,2)$ are in the same coset. Therefore $c_2=c_3$.\\
Case 4: Suppose $(1,1)+B=-(1,0)+B$. Then $(2,1)\in B$, so $(5,3)+B=(1,1)+B$. This shows that $(1,1)$,$-(1,0)$, and $-(3,1)$ are in the same coset. Therefore $c_2=c_3$.\\
Case 5: Suppose $(1,1)+B=(1,2)+B$. Then $(0,1)\in B$, so $(1,0)+B=(1,1)+B$. This shows $(1,1)$,$(1,0)$, and $(1,2)$ are in the same coset. Therefore $c_2=c_3$.\\
Case 6: Suppose $(1,1)+B=-(1,2)+B$. Then $(2,3)\in B$, which implies that $(2,1)\in B$. So $(5,3)+B=(1,1)+B$. This shows that $(1,1)$,$-(1,2)$, and $-(3,1)$ are in the same coset. Therefore $c_2=c_3$.\\
This completes the proof.\end{proof}

\newpage
\section{Summary}
The table below shows the findings of our work. The groups $\Z_{6}\oplus \Z_{2}$ and $\Z_{10}\oplus \Z_{2}$ are of the form $\Z_{2p}\oplus\Z_{2}$ where $p$ is a prime number, so by Theorem $\ref{thm:21}$ these groups do not contain nonseparating subsets. On the other hand, Dr. Edgar Saenz proved that groups of the form $\Z_{2p}\oplus\Z_{2p}$ -where $p$ is a prime number- cannot contain nonseparating subsets, for further details see Chapter 3 of \cite{S}. Hence, $\Z_{10}\oplus\Z_{10}$ does not contain any nonseparating subset.\\
%\vspace{.1in}\\
\begin{center}
\begin{tabular}{|c|c|c|c|}
\hline
Group&Hurwitz class representative& cardinality of the class& Type\\
\hline
$\Z_{2}\oplus \Z_{2}$&NONE&0&-\\
\hline
$\Z_{4}\oplus \Z_{2}$&$\{(1,0),(0,1),(1,1),(2,1)\}$&$2$&$3$\\
\hline
$\Z_{6}\oplus \Z_{2}$&NONE&$0$&- \\
\hline
$\Z_{4}\oplus \Z_{4}$&$\{(0,0),(1,0),(1,2),(2,0)\}$&$6$&$3$ \\
\hline
$\Z_{4}\oplus \Z_{4}$&$\{(0,1),(1,0),(1,2),(2,1)\}$&$3$&$2$\\
\hline
$\Z_{4}\oplus \Z_{4}$&$\{(1,0),(0,1),(1,1),(3,1)\}$&$12$& Lemma 3.2\\
\hline
$\Z_{6}\oplus \Z_{6}$&$\{(0,2),(1,0),(1,1),(2,1)\}$&$24$& Exceptional\\
\hline
$\Z_{6}\oplus \Z_{6}$&$\{(0,1),(1,2),(1,4),(2,3)\}$&$36$& Exceptional\\
\hline
$\Z_{6}\oplus \Z_{6}$&$\{(0,1),(2,1),(2,3),(2,5)\}$&$4$& Exceptional\\
\hline
$\Z_{8}\oplus \Z_{2}$&$\{(2,0),(0,1),(2,1),(4,1)\}$&$2$&$3$\\
\hline
$\Z_{8}\oplus \Z_{2}$&$\{(1,0),(2,0),(3,0),(2,1)\}$&$2$&$1$\\
\hline
$\Z_{8}\oplus\Z_{4}$&$\{(0,0),(2,0),(4,0),(2,2)\}$&$2$&$3$\\
\hline
$\Z_{8}\oplus\Z_{4}$&$\{(0,0),(0,1),(4,1),(0,2)\}$&$4$&$3$\\
\hline
$\Z_{8}\oplus\Z_{4}$&$\{(1,0),(2,0),(3,0),(2,2)\}$&$4$&$1$\\
\hline
$\Z_{8}\oplus\Z_{4}$&$\{(1,1),(2,1),(3,1),(6,1)\}$&$8$&$2$\\
\hline
$\Z_{8}\oplus\Z_{4}$&$\{(1,0),(1,1),(1,2),(3,1)\}$&$8$& Exceptional\\
\hline
$\Z_{8}\oplus\Z_{4}$&$\{(2,0),(0,1),(2,1),(4,1)\}$&$8$& Lemma 3.2\\
\hline
$\Z_{8}\oplus\Z_{4}$&$\{(2,0),(0,1),(6,1),(2,2)\}$&$4$& Lemma 3.2 \\
\hline
$\Z_{8}\oplus\Z_{4}$&$\{(2,0),(0,1),(4,1),(2,2)\}$&$2$&$2$\\
\hline
$\Z_{8}\oplus\Z_{4}$&$\{(0,1),(2,1),(4,1),(6,1)\}$&$1$&$2$\\
\hline
$\Z_{10}\oplus \Z_{2}$& NONE &$0$&-\\
\hline
$\Z_{10}\oplus \Z_{10}$& NONE &$0$&-\\
\hline
\end{tabular}
\end{center}\
\\
The interested reader can verify the above information in \cite{W}.
\newpage
%%%%%%%%%%%%%%%%%%%%%%%%%%%%%%%%%%%%%%%%%%%%%%%%%%%%%%%%%%%%%%%%%%%%%%%%%%%%%%%%%%%%%%%%%%%%%%%%%%%%%%%%%%%%%%%%%%%%%%%%%%%%%%%%%%%%%%%%%%%%%%%%%%%%

\maketitle\appendix
\section{Exceptional nonseparating subsets of $A=\Z_{6}\oplus\Z_{6}$}\
\\
We use the computer program to verify that the sets $H_{2}=\{\pm(1,0),\pm(0,1),\pm(5,1),\pm(2,2)\}$ and $H_{3}=\{\pm(1,0),\pm(0,1),\pm(5,4),\pm(2,5)\}$ are nonseparating subsets of $\Z_{6}\oplus\Z_{6}$.\\
\\
The following data corresponds to $H_{2}=\{\pm(1,0),\pm(0,1),\pm(5,1),\pm(2,2)\}$:\\
\\
\begin{tabular}{|c|c|c|}
\hline
$B$&generator of $A/B$&coset numbers\\
\hline
$\langle(0,1)\rangle$&$(1,4)+B$&$0,1,1,2$\\
\hline
$\langle(0,1)\rangle$&$(5,4)+B$&$0,1,1,2$\\
\hline
$\langle(1,0)\rangle$&$(4,1)+B$&$0,1,1,2$\\
\hline
$\langle(1,0)\rangle$&$(1,5)+B$&$0,1,1,2$\\
\hline
$\langle(1,1)\rangle$&$(0,1)+B$&$0,1,1,2$\\
\hline
$\langle(1,1)\rangle$&$(5,4)+B$&$0,1,1,2$\\
\hline
$\langle(1,2)\rangle$&$(5,5)+B$&$1,2,2,3$\\
\hline
$\langle(1,2)\rangle$&$(4,1)+B$&$1,2,2,3$\\
\hline
$\langle(1,3)\rangle$&$(1,4)+B$&$1,2,2,3$\\
\hline
$\langle(1,3)\rangle$&$(4,5)+B$&$1,2,2,3$\\
\hline
$\langle(1,4)\rangle$&$(1,5)+B$&$0,1,1,2$\\
\hline
$\langle(1,4)\rangle$&$(4,3)+B$&$0,1,1,2$\\
\hline
\end{tabular}
\quad
\begin{tabular}{|c|c|c|}
\hline
$B$&generator of $A/B$&coset numbers\\
\hline
$\langle(1,5)\rangle$&$(0,1)+B$&$0,1,1,2$\\
\hline
$\langle(1,5)\rangle$&$(1,4)+B$&$0,1,1,2$\\
\hline
$\langle(2,1)\rangle$&$(5,5)+B$&$1,2,2,3$\\
\hline
$\langle(2,1)\rangle$&$(1,4)+B$&$1,2,2,3$\\
\hline
$\langle(2,3)\rangle$&$(1,4)+B$&$1,2,2,3$\\
\hline
$\langle(2,3)\rangle$&$(1,5)+B$&$1,2,2,3$\\
\hline
$\langle(2,5)\rangle$&$(5,4)+B$&$0,1,1,2$\\
\hline
$\langle(2,5)\rangle$&$(1,5)+B$&$0,1,1,2$\\
\hline
$\langle(3,1)\rangle$&$(1,4)+B$&$1,2,2,3$\\
\hline
$\langle(3,1)\rangle$&$(5,4)+B$&$1,2,2,3$\\
\hline
$\langle(3,2)\rangle$&$(1,5)+B$&$1,2,2,3$\\
\hline
$\langle(3,2)\rangle$&$(5,5)+B$&$1,2,2,3$\\
\hline
\end{tabular}\vspace{.2in}\\
\\
The following data corresponds to $H_{3}=\{\pm(1,0),\pm(0,1),\pm(5,4),\pm(2,5)\}$:\\
\\
\begin{tabular}{|c|c|c|}
\hline
$B$&generator of $A/B$&coset numbers\\
\hline
$\langle(0,1)\rangle$&$(1,4)+B$&$0,1,1,2$\\
\hline
$\langle(0,1)\rangle$&$(5,4)+B$&$0,1,1,2$\\
\hline
$\langle(1,0)\rangle$&$(4,1)+B$&$0,1,1,2$\\
\hline
$\langle(1,0)\rangle$&$(1,5)+B$&$0,1,1,2$\\
\hline
$\langle(1,1)\rangle$&$(0,1)+B$&$1,1,1,3$\\
\hline
$\langle(1,1)\rangle$&$(5,4)+B$&$1,1,1,3$\\
\hline
$\langle(1,2)\rangle$&$(5,5)+B$&$0,1,1,2$\\
\hline
$\langle(1,2)\rangle$&$(4,1)+B$&$0,1,1,2$\\
\hline
$\langle(1,3)\rangle$&$(1,4)+B$&$1,1,1,3$\\
\hline
$\langle(1,3)\rangle$&$(4,5)+B$&$1,1,1,3$\\
\hline
$\langle(1,4)\rangle$&$(1,5)+B$&$1,2,2,3$\\
\hline
$\langle(1,4)\rangle$&$(4,3)+B$&$1,2,2,3$\\
\hline
\end{tabular}
\quad
\begin{tabular}{|c|c|c|}
\hline
$B$&generator of $A/B$&coset numbers\\
\hline
$\langle(1,5)\rangle$&$(0,1)+B$&$1,1,1,3$\\
\hline
$\langle(1,5)\rangle$&$(1,4)+B$&$1,1,1,3$\\
\hline
$\langle(2,1)\rangle$&$(5,5)+B$&$1,2,2,3$\\
\hline
$\langle(2,1)\rangle$&$(1,4)+B$&$1,2,2,3$\\
\hline
$\langle(2,3)\rangle$&$(1,4)+B$&$1,2,2,3$\\
\hline
$\langle(2,3)\rangle$&$(1,5)+B$&$1,2,2,3$\\
\hline
$\langle(2,5)\rangle$&$(5,4)+B$&$0,1,1,2$\\
\hline
$\langle(2,5)\rangle$&$(1,5)+B$&$0,1,1,2$\\
\hline
$\langle(3,1)\rangle$&$(1,4)+B$&$1,1,1,3$\\
\hline
$\langle(3,1)\rangle$&$(5,4)+B$&$1,1,1,3$\\
\hline
$\langle(3,2)\rangle$&$(1,5)+B$&$1,2,2,3$\\
\hline
$\langle(3,2)\rangle$&$(5,5)+B$&$1,2,2,3$\\
\hline
\end{tabular}\\
\\
\begin{center}
List of Symbols.
\end{center}
$\Z^{+}$ is the set of positive integers.\\
$S^{2}$ topological 2-sphere.\\
$\Z^{2}$ is the 2-dimensional integer lattice.\\ 
$P_{f}$ is postcritical set of the Thurston map $f$.\\
$|A|$ is the order of the group $A$.\\
$o(g)$ is the order of the element $g$.\\

\begin{thebibliography}{99}
\bibitem{BEKP} X. Buff, A. Epstein, S. Koch, and K. Pilgrim,\begin{em} On Thurston's pullback map\end{em}. In $Complex$ $dynamics$, pp 561--583. AK Peters, Wellesley (2009). MR 2508269 (2010g:37071)

%\bibitem{CHDR}J. Hillar, L. Rhea, \begin{em} Automorphisms Of Finite Abelian Groups\end{em}. ArXiv:math/0605185 \textbf{[math.GR]}.
\bibitem{CFPP} J.W. Cannon, W.J. Floyd, W.R. Parry, and K.M. Pilgrim,\begin{em} Nearly Euclidean Thurston maps and finite subdivision rules\end{em}. Conform. Geom. Dyn. \textbf{16} (2012), 209--255(electronic).

\bibitem{KLS}  K. Pilgrim, W. Floyd, G. Kelsey,
S. Koch, R. Lodge, W. Parry, and Edgar A. Saenz. \begin{em} Origami, affine maps and complex dynamics \end{em}. Arnold Math. J. September
2017, Volume 3, Issue 3, pp 365-395.

\bibitem{S}
E. A. Saenz Maldonado,\begin{em}On Nearly Euclidean Thurston maps,\end{em} Ph.D. Thesis, Virginia Tech, 2012.

\bibitem{W} 
W. Li \begin{em}https://github.com/weitao92/NET-MAP-non$_{-}$separating$_{-}$sets-tracker\end{em}

\end{thebibliography}
\end{document}



https://github.com/weitao92/NET-MAP-non$_{-}$separating$_{-}$sets-tracker








































\textbf{Counting elements in each equivalence class.}\vspace{.1in}\\
We know that 
$$H_{1}=\{\pm(0,2),\pm(2,0),\pm(2,4),\pm(2,2)\}$$
$$H_{2}=\{\pm(1,0),\pm(0,1),\pm(5,1),\pm(2,2)\}$$
$$H_{3}=\{\pm(1,0),\pm(0,1),\pm(5,4),\pm(2,5)\}$$
are nonseparating subsets of $A$.\\

\begin{lem}\label{lem:102}  $H_{1}, H_{2}, H_{3}$ are in distinct equivalence classes.\end{lem} 
\begin{proof} We first identify the group $A$ with $\Z_{2}\times \Z_{2}\times \Z_{3}\times \Z_{3}$ by using the canonical isomorphism $f:\Z_{6}\to \Z_{2}\oplus\Z_{3}$ defined by $f(\ov{1})=(\ov{1},\ov{1})$. Then the sets $H_{1}$, $H_{2}$ and $H_{3}$ are identified with the following sets
$$H_{1}=\{\pm(0,0,1,0),\pm(0,0,0,1),\pm(0,0,2,1),\pm(0,0,2,2)\}$$
$$H_{2}=\{\pm(1,0,1,0),\pm(0,1,0,1),\pm(1,1,2,1),\pm(0,0,2,2)\}$$
$$H_{3}=\{\pm(1,0,1,0),\pm(0,1,0,1),\pm(1,0,2,1),\pm(0,1,2,2)\}$$

Let $L=(F,G)$ be an automorphism of $\Z_{2}\times \Z_{2}\times \Z_{3}\times \Z_{3}$. Then $L(x,y)=(F(x),G(y))$ where $F\in GL_{2}(\mathbb{F}_{2})$ and $G\in GL_{2}(\mathbb{F}_{3})$. Let $b$ be an element of $\Z_{2}\times \Z_{2}$ and define $T(x,y)=L(x,y)+(b,0)$. We apply $T$ to the elements of $H_{2}$ and obtain  
$$T(1,0,1,0)=(F(1,0)+b,G(1,0))\hspace{1in} T(0,1,0,1)=(F(0,1)+b,G(0,1))$$
$$T(1,1,2,1)=(F(1,1)+b,G(2,1))\hspace{1in} T(0,0,2,2)=(F(0,0)+b,G(2,2))$$

Since $x\mapsto F(x)+b$ is an affine mapping on $\Z_{2}\times \Z_{2}$, it will permute the elements of $\Z_{2}\times \Z_{2}$; so $H_{1}$ and $H_{3}$ cannot be in the class of $H_{2}$.

We now apply $T$ to the elements of $H_{3}$ and obtain  
$$T(1,0,1,0)=(F(1,0)+b,G(1,0))\hspace{1in} T(0,1,0,1)=(F(0,1)+b,G(0,1))$$
$$T(1,0,2,1)=(F(1,0)+b,G(2,1))\hspace{1in} T(0,0,2,2)=(F(0,1)+b,G(2,2))$$
So the first coordinate of these elements can only be $F(0,1)+b$ and $F(0,1)+b$. Since $F$ is a bijection we are getting exactly two distinct elements of $\Z_{2}\oplus\Z_{2}$ (two out of four). Therefore, $H_{1}$ and $H_{2}$ cannot be in the equivalence class of $H_{3}$ and this completes the proof.\end{proof}


\begin{lem}\label{lem:103} The class of $H_{1}$ has exactly 4 elements, the class of $H_{2}$ has exactly 24 elements and the class of $H_{3}$ has exactly 36 elements.\end{lem}

\begin{proof} Since the mapping $G$ (see proof above) is an element of $GL_{2}(\mathbb{F}_{3})$, it is a bijection on $\Z_{3}\times \Z_{3}$ fixing $(0,0)$. So the number of elements on each class depends on the mappings of the form $x\mapsto F(x)+b$ and the desired outputs on the first entry. The number of elements in the class of $H_{1}$ only depends on the translation term $b$, whence the class of $H_{1}$ has exactly 4 elements. Since $x\mapsto F(x)+b$ permutes the elements of $\Z_{2}\times \Z_{2}$, the class of $H_{2}$ has 4! elements. Finally, we only want two distinct outputs for the first entry of the elements of the nonseparating sets in the class of $H_{3}$, we can identify these outputs as the range of the affine non-constant mappings on $\Z_{2}\times \Z_{2}$ defined by $x\mapsto Px+b$ where $P$ is not invertible and not equal to the zero matrix, and $b$ is any element of $\Z_{2}\times \Z_{2}$. There are 9 choices for the matrix $P$ and 4 choices for the element $b$. Thus, the class of $H_{3}$ has exactly 36 elements.\end{proof}

Therefore, there are exactly three distinct equivalence classes of nonseparating sets in $\Z_{6}\times \Z_{6}$ and my program verifies the number of distinct classes and the number of nonseparating sets in each class.\\




{\blue need to be reword/ cite Walter 
Let $A=\Z_{m}\oplus \Z_{n}$, where $m$ and $n$ are even positive integers with $n|m$ and $m\geq4$. Lemma \ref{lem:60} leads to 3 choices for $a$,$b$,$c$ and $d$. Let $m$ and $n$ be even positive integers with $m$ divisible by $4$. Then we have the following 3 types of nonseparating sets:
\textbf{TYPE 1} $m \ne 4$, $n = 2, a = (\frac{m}{4}, 0), b = (\frac{m}{4}, 1), c = (1, 0), d = (\frac{m}{2} + 1, 0) \in \Z_{m}\oplus \Z_{2}$\\
\textbf{TYPE 2} $n = 4$, $a=(0,1)$, $b=(\frac{m}{2}, 1)$, $c=(1,0)$, $d=(1, 2)\in\Z_{m}\oplus \Z_{4}$.\\
\textbf{TYPE 3} $m\ne8$ $n=2$, $a=(\frac{m}{4}, 0)$, $b=(\frac{m}{4},1)$, $c=(2, 0)$, $d=(\frac{m}{2}+2, 0)\in\Z_{m}\oplus \Z_{2}$.\\
These 3 types of nonseparating subsets provide infinitely many equivalence classes as m varies. In addition to these equivalence classes there are 5 exceptional equivalence classes contained in groups with exponents m at most 8 which i will describe in the next section.}











\begin{prop}\label{prop:a1} Let $G$ be a finite cyclic group of order $n$ and let $h$ be an element of order $m$ in $G$. Then there exists $g\in G$ such that $\langle g\rangle=G$ and $g^{n/m}=h$.
\end{prop}
\begin{proof}
Choose $a\in G$ such that $\langle a\rangle=G$. Then $\langle a^{n/m}\rangle=\langle h\rangle$, so there exists $r\in\n$ such that $a^{nr/m}=h$; of course $\gcd(r,m)=1$. Also if $d\in\n$,
then $a^{nd/m}=h$ if and only if $d\equiv r\mod m$. Let $q$ be the product of the primes which divide $n/m$ but do not divide $m$. Thus, $\gcd(m,q)=1$ and $\gcd(s,n)=1$ if and only if $\gcd(s,m)=\gcd(s,q)=1$. By the Chinese remainder theorem, we may choose $\tau\in\n$ such that $\tau\equiv r \mod m$ and $\tau\equiv 1\mod q$. 
Then $\gcd(\tau,n)=1$, so $a^{\tau}$ generates $G$. Also $(a^{\tau})^{n/m}=h$, as required.
\end{proof}

\begin{prop}\label{prop:a2} Every element of $\mathbb{Z}/n\mathbb{Z}\oplus\mathbb{Z}/n\mathbb{Z}$ is a multiple of a basis element.
\end{prop}
\begin{proof}
In fact, let $g=(\ov{x},\ov{y})\in\mathbb{Z}/n\mathbb{Z}\oplus\mathbb{Z}/n\mathbb{Z}$. If $\ov{x}=\ov{0}$ or $\ov{y}=\ov{0}$, the proposition follows. Assume $\ov{x}\neq\ov{0}$ and $\ov{y}\neq\ov{0}$, then $g=d(\ov{x/d},\ov{y/d})$ where $d=\gcd(x,y)$. Since $\gcd(x/d,y/d)=1$, $(\ov{x/d},\ov{y/d})$ is a basis element and the proposition follows. 
\end{proof}

\begin{prop}\label{prop:a3} Let $A=\mathbb{Z}/n\mathbb{Z}\oplus\mathbb{Z}/n\mathbb{Z}$. Let $A'$ be a subgroup of $A$, and let $B'$ be a cyclic subgroup of $A'$ so that $A'/B'$ is also cyclic. Then there exists a cyclic subgroup $B$ of $A$ such that $A/B$ is cyclic and $A'\cap B = B'$.
\end{prop}
\begin{proof} Since $A$ is the direct sum of its Sylow subgroups and each Sylow subgroup of $A$ is isomorphic to a direct sum of the form $\mathbb{Z}/p^{r}\mathbb{Z}\oplus\mathbb{Z}/p^{r}\mathbb{Z}$ for some prime $p$, it suffices to prove the proposition for $n=p^{r}$ where $p$ is a prime number and $r\in\Z^{+}$.

\text{First Case.} Suppose that $A'$ is cyclic. By Proposition \ref{prop:a2}, every element of $A'$ is a multiple of a basis element of $A$. Let $v\in A$ be a basis element such that $\left< v \right>$ contains $A'$. Choose $w\in A$ so that $\{v,w\}$ is a basis for $A$. Let $\varphi$ be the automorphism of $A$ defined on these generators by $\varphi(v)=(1,0)$ and $\varphi(w)=(0,1)$. Let $b'$ be a generator of $B'$ and let $k=o(b')$. Since $\varphi(b')\in\left<(1,0)\right>$, by Proposition \ref{prop:a1}, there exists $g\in\mathbb{Z}/n\mathbb{Z}$ such that $\langle g\rangle=\mathbb{Z}/n\mathbb{Z}$ and $\varphi(b')=(n/k)(g,0)$. Let $T$ be the automorphism of $A$ defined on generators by $T(g,0)=(1,0)$ and $T(0,1)=(0,1)$. Now consider the isomorphism $f=T\circ\varphi$. Then $A'$ is isomorphic to $f(A')$ which is a subgroup contained in the subgroup $\left<(1,0)\right>$. Let  $b\in A$ so that $f(b)=(1,k)$. Using these coordinates, we have $\left<f(b)\right>\cap f(A')=\left<(n/k,0)\right>$. Then $\left<b\right>\cap A'=\left<b'\right>$. Finally, set $B=\left<b\right>$. Since  $o(b)=n$, $A/B$ is cyclic as required.

\text{Second Case.} Suppose that $A'$ is not cyclic. In this case, there are positive integers $s, t$ so that $A'\cong\mathbb{Z}/p^{s}\mathbb{Z}\oplus\mathbb{Z}/p^{t}\mathbb{Z}$. We first show that no cyclic subgroup of $A'$ properly contains $B'$. If $C$ is a cyclic subgroup of $A'$ such that $B'\subsetneq C$, then $B'\subseteq pC$ and so $B'\subseteq pA'\subsetneq A'$. The Third Isomorphism Theorem for groups would then imply that the quotient group $(A'/B')/(pA'/B')$ is isomorphic to $A'/pA'$. This is a contradiction because $A'/B'$ is cyclic and $A'/pA'$ is not. Now let $b'$ be a generator of $B'$ and let $b$ be a basis element of $A$ such that $\left<b\right>$ contains $B'$. Set $B=\left<b\right>$.  Because $b$ is a basis element of $A$, $A/B$ is cyclic. Finally, since $A'\cap B$ is a cyclic subgroup of $A'$ that contains $B'$, we have $A'\cap B=B'$.\end{proof}

\begin{prop}\label{prop:a01} Let $A$ be a finite Abelian group generated by two elements. Let $A'$ be a subgroup of $A$, and let $B'$ be a cyclic subgroup of $A'$ so that $A'/B'$ is also cyclic. Then there exists a cyclic subgroup $B$ of $A$ such that $A/B$ is cyclic and $A'\cap B = B'$.\end{prop}
\begin{proof} We may assume that $A=\mathbb{Z}/m\mathbb{Z}\oplus\mathbb{Z}/n\mathbb{Z}$ where $m$ and $n$ are positive integers such that $m|n$. Let $i_{c}:A\to\mathbb{Z}/n\mathbb{Z}\oplus\mathbb{Z}/n\mathbb{Z}$ be the canonical monomorphism defined on generators by $i_{c}(1,0)=(n/m,0)$ and $i_{c}(0,1)=(0,1)$. Then $A$ is isomorphic to the subgroup $\langle n/m\rangle\oplus\mathbb{Z}/n\mathbb{Z}$. So, without loss of generality, we may assume that $A$ is a subgroup of $\mathbb{Z}/n\mathbb{Z}\oplus\mathbb{Z}/n\mathbb{Z}$. Now let $A'$ be a subgroup of $A$, and let $B'$ be a cyclic subgroup of $A'$ so that $A'/B'$ is cyclic. By Proposition \ref{prop:a3}, there exists a cyclic subgroup $B$ of $\mathbb{Z}/n\mathbb{Z}\oplus\mathbb{Z}/n\mathbb{Z}$ such that $(\mathbb{Z}/n\mathbb{Z}\oplus\mathbb{Z}/n\mathbb{Z})/B$ is cyclic and $A'\cap B=B'$. Now, set $\tilde{B}=A\cap B$. Obviously $\tilde{B}$ is a cyclic subgroup of $A$ and $A'\cap \tilde{B}=A'\cap(A\cap B)=(A'\cap A)\cap B=A'\cap B=B'$. 
Finally, using the canonical projection $\phi:A\to(\mathbb{Z}/n\mathbb{Z}\oplus\mathbb{Z}/n\mathbb{Z})/B$ defined by $x\mapsto x+B$, one sees that $\text{Ker}(\phi)=A\cap B=\tilde{B}$ and so $A/\tilde{B}$ is isomorphic to $\phi(A)$. Since $(\mathbb{Z}/n\mathbb{Z}\oplus\mathbb{Z}/n\mathbb{Z})/B$ is cyclic, $\phi(A)$ also is and therefore $A/\tilde{B}$ is cyclic.\end{proof}




One curious fact is that despite the first class coming from $\Z_{2}\times \Z_{4}$ and the second class coming from  $\Z_{2}\times \langle(2)\rangle$ which is $\cong$ $\Z_{2}\times \Z_{4}$, but there is no automorphism in  $\Z_{4}\times \Z_{8}$ that maps them to each other, here is a brief proof:

\begin{cor}$A=\Z_{4}\times \Z_{8}$, and let $\psi$: $A \mapsto A$ be a composition of automorphism in $A$ and translation by element of order 2. Let $H_1$ = $\{(0,0)(1,0)(2,0)(1,4)\}$, and let $H_2$ = $\{(0,2)(2,0)(2,2)(2,4)\}$. Then there exist no $\psi$ such that $\psi(H_1) = H_2$.\end{cor}

\begin{proof} Let's first assume $\psi$ is translated by b = $(2,0)$, then since automorphism preserves order of elements and then

case 1:

$(0,0)$ by automorphism maps to $(0,0)$ translate by b maps to $(2,0)$

$(2,0)$ by automorphism maps to $(0,4)$ translate by b maps to $(2,4)$

$(1,0)$ by automorphism maps to $(0,2)$ translate by b maps to $(2,2)$

$(1,4)$ by automorphism maps to $(2,2)$ translate by b maps to $(0,2)$

Then $(0,4)$ by automorphism maps to $(2,0)$, but $(0,4) = 4(0,1)$, and $auto(4(0,1)) = 4auto((0,1))$, let's assume $auto((0,1))$ maps to $(a,b)$, then $auto((0,4))$ maps to $(4a,4b)$, but since $4a$ cannot be 2 in $Z_4$, therefore we are having a contradiction.\\

case 2:

$(0,0)$ by automorphism maps to $(0,0)$ translate by b maps to $(2,0)$

$(2,0)$ by automorphism maps to $(0,4)$ translate by b maps to $(2,4)$

$(1,0)$ by automorphism maps to $(2,2)$ translate by b maps to $(0,2)$

$(1,4)$ by automorphism maps to $(0,2)$ translate by b maps to $(2,2)$

Then $(0,4)$ by automorphism maps to (2,0) again, and following case 1 we will arrive to the same contradiction.\\

Follow the same logic and chose different b such that $o(b) = 2$, we will prove there is no automorphism and translation by element of order 2 that will map the first 2 classes to each other.\end{proof}

We will now focus on the $\{(0,1)(1,3)(1,5)(2,3)\}$ which is the exceptional nonseparating set in $\Z_{4}\times \Z_{8}$:

\newpage







\textbf{Remarks on $GL_{2}(\mathbb{F}_{2})$}\\
The group $GL_{2}(\mathbb{F}_{2})$ contains exactly six elements. These are 
$$\left[{\begin{array}{*{20}c}
    1 && 0 \vspace{.05in}\\
     0 && 1
 \end{array}}\right],\hspace{1in}\\
 \left[{\begin{array}{*{20}c}
     1 && 1 \vspace{.05in}\\
     0 && 1
 \end{array}}\right],\left[{\begin{array}{*{20}c}
    0 && 1 \vspace{.05in}\\
    1 && 1
 \end{array}}\right],\left[{\begin{array}{*{20}c}
    0 && 1 \vspace{.05in}\\
    1 && 0
 \end{array}}\right],\hspace{1in}\\
 \left[{\begin{array}{*{20}c}
    1 && 1 \vspace{.05in}\\
    1 && 0
 \end{array}}\right],
\left[{\begin{array}{*{20}c}
    1 && 0 \vspace{.05in}\\
    1 && 1
 \end{array}}\right]$$
Denote these matrices by 
$$M=\left[{\begin{array}{*{20}c}
     m_{11} && m_{12} \vspace{.05in}\\
     m_{21} && m_{22}
 \end{array}}\right]$$

Each matrix $M$ determines 16 automorphisms of $\Z_{4}\times \Z_{4}$ as follows. A matrix with integer coefficients $$\left[{\begin{array}{*{20}c}
    a_{11} && a_{12} \vspace{.05in}\\
     a_{21} && a_{22}
 \end{array}}\right]$$
induces an automorphism on $\Z_{4}\times \Z_{4}$ if and only if $a_{ij}\in\{0,1,2,3\}$ and $a_{ij}\equiv m_{ij} (\text{mod}\ 2)$. For instance, if we consider 
$$M=\left[{\begin{array}{*{20}c}
     1 && 1 \vspace{.05in}\\
     0 && 1
 \end{array}}\right],$$
the corresponding automorphisms must verify the following $a_{11}\equiv 1 (\text{mod}\ 2)$; $a_{12}\equiv 1 (\text{mod}\ 2)$; $a_{21}\equiv 0 (\text{mod}\ 2)$; and $a_{22}\equiv 1 (\text{mod}\ 2)$. Since $a_{ij}\in\{0,1,2,3\}$, we have $a_{11}\in\{1,3\}$; $a_{1,2}\in\{1,3\}$; $a_{21}\in\{0,2\}$; and $a_{22}\in\{1,3\}$. This is how we get the 16 automorphisms associated to $M$. Since $GL_{2}(\mathbb{F}_{2})$ contains 6 distinct matrices $M$, there are exactly 96 automorphisms on $\Z_{4}\times \Z_{4}$.\vspace{.4in}\\

\textbf{Remarks on $GL_{2}(\mathbb{F}_{3})$}\\
The group $GL_{2}(\mathbb{F}_{3})$ contains exactly 48 elements. These are mappings are precisely the automorphisms of $\Z_{3}\times \Z_{3}$. Below is a short description. A matrix with integer coefficients $$M=\left[{\begin{array}{*{20}c}
    m_{11} && m_{12} \vspace{.05in}\\
    m_{21} && m_{22}
 \end{array}}\right]$$
induces an automorphism on $\Z_{3}\times \Z_{3}$ if and only if $m_{ij}\in{0,1,2}$ and the columns of $M$ form a basis for $\mathbb{F}_{3}\times \mathbb{F}_{3}$.\\
For details of the generalized matrix form of Automorphisms of finite Abelian groups, see \cite{CHDR}

